 \section{Cable}
 
 %element finis
 %jaulin towing
 It exist a certain number of different method to simulate a cable from finite elements to lumped mass
 with different implementations in each cases, most off the literature also will present a multi systems
 problem 
 
 \section{Sailboat model}

The model of the sailboat come from~\cite{Melin2016} a modified version of the model in~\cite{LeBars2013}.



\begin{figure}[H]
\centering
\psscalebox{0.45 0.45} % Change this value to rescale the drawing.
{
% \usepackage[usenames,dvipsnames]{pstricks}
% \usepackage{epsfig}
% \usepackage{pst-grad} % For gradients
% \usepackage{pst-plot} % For axes
% \usepackage[space]{grffile} % For spaces in paths
% \usepackage{etoolbox} % For spaces in paths
% \makeatletter % For spaces in paths
% \patchcmd\Gread@eps{\@inputcheck#1 }{\@inputcheck"#1"\relax}{}{}
% \makeatother
% % User Packages:
% \usepackage{amsmath}
% \usepackage{amsfonts}
% \usepackage{amssymb}
% \usepackage{algorithm}
% \usepackage{algorithmic}
\psscalebox{1.0 1.0} % Change this value to rescale the drawing.
{
\begin{pspicture}(0,-6.3229065)(27.976694,6.3229065)
\definecolor{colour6}{rgb}{0.627451,0.627451,0.627451}
\definecolor{colour7}{rgb}{0.40392157,0.3882353,0.3882353}
\pspolygon[linecolor=black, linewidth=0.04, fillstyle=solid,fillcolor=black](4.2265778,0.6801973)(4.2265778,0.6687973)(4.1737776,0.6687973)(4.1737776,0.6801973)
\pspolygon[linecolor=black, linewidth=0.04, fillstyle=solid,fillcolor=black](4.2265778,0.6801973)(4.2265778,0.6687973)(4.1737776,0.6687973)(4.1737776,0.6801973)
\psline[linecolor=black, linewidth=0.02](4.494638,0.92061454)(4.494638,-5.747811)
\psline[linecolor=black, linewidth=0.02](4.5202856,0.92916393)(0.6303709,0.92916393)
\psline[linecolor=black, linewidth=0.04, arrowsize=0.05291667cm 2.0,arrowlength=1.4,arrowinset=0.0]{->}(0.65601873,-5.755437)(12.642086,-5.755437)
\psline[linecolor=black, linewidth=0.04, arrowsize=0.05291667cm 2.0,arrowlength=1.4,arrowinset=0.0]{->}(0.63892037,-5.755437)(0.63892037,5.409901)
\psline[linecolor=colour6, linewidth=0.03, linestyle=dashed, dash=0.17638889cm 0.10583334cm](0.65783095,-5.7774105)(6.7124696,4.760216)
\psline[linecolor=black, linewidth=0.020657333](1.7846738,-0.7115013)(4.5205197,-2.0897024)
\psline[linecolor=black, linewidth=0.02](5.956275,3.439544)(4.5137773,-2.090031)
\psline[linecolor=black, linewidth=0.02](5.956275,3.446615)(1.7843451,-0.72531486)
\rput[bl](4.325926,-6.3229065){{\huge $x$}}
\rput[bl](0.0,0.5215381){{\huge $y$}}
\psarc[linecolor=colour7, linewidth=0.03, linestyle=dashed, dash=0.17638889cm 0.10583334cm, dimen=inner, arrowsize=0.05291667cm 2.0,arrowlength=1.4,arrowinset=0.0]{->}(4.4962683,0.90230674){0.44}{19.308674}{310.4}
\rput[bl](4.44457,1.4879671){{\Large $w$}}
\psarc[linecolor=black, linewidth=0.04, linestyle=dashed, dash=0.17638889cm 0.10583334cm, dimen=outer, arrowsize=0.05291667cm 2.0,arrowlength=1.4,arrowinset=0.0]{->}(0.67470086,-5.742382){1.16}{0.0}{58.991932}
\rput[bl](1.8318518,-4.7999434){{\LARGE $\theta$}}
\rput[bl](0.8096296,5.5111675){{\huge North }}
\rput[bl](10.942963,-5.5110545){{\huge East}}
\rput[bl](16.698519,-5.103647){{\LARGE $\theta$}}
\rput[bl](23.935556,-3.555499){{\LARGE $\psi_{tw}$}}
\rput[bl](17.683704,1.4222788){{\LARGE $\psi_{aw}$}}
\psarc[linecolor=black, linewidth=0.03, linestyle=dashed, dash=0.17638889cm 0.10583334cm, dimen=outer, arrowsize=0.05291667cm 2.0,arrowlength=1.4,arrowinset=0.0]{->}(18.638119,-0.08968701){1.16}{60.0}{190.0}
\psarc[linecolor=black, linewidth=0.03, linestyle=dashed, dash=0.17638889cm 0.10583334cm, dimen=outer, arrowsize=0.05291667cm 2.0,arrowlength=1.4,arrowinset=0.0]{->}(23.642906,-5.004217){1.16}{0.0}{127.42864}
\rput[bl](21.683704,-2.086553){{\LARGE $a_{tw}$}}
\rput[bl](22.29624,5.598347){{\LARGE $v$}}
\rput[bl](13.711624,0.35219333){{\LARGE $-v$}}
\rput[bl](14.480855,-1.3093451){{\LARGE $a_{aw}$}}
\rput[bl](16.650085,2.5675778){{\LARGE $a_{tw}$}}
\psarc[linecolor=black, linewidth=0.04, linestyle=dashed, dash=0.17638889cm 0.10583334cm, dimen=outer, arrowsize=0.05291667cm 2.0,arrowlength=1.4,arrowinset=0.0]{->}(15.365983,-5.727294){1.16}{0.0}{58.991932}
\psline[linecolor=colour6, linewidth=0.03, linestyle=dashed, dash=0.17638889cm 0.10583334cm](23.691505,-5.0271945)(27.976694,-5.0271945)
\psline[linecolor=black, linewidth=0.03, arrowsize=0.05291667cm 2.0,arrowlength=1.4,arrowinset=0.0]{->}(23.6699,-5.027194)(21.300167,-2.0771163)
\psline[linecolor=black, linewidth=0.02](20.27553,2.7457647)(16.1036,-1.4261652)
\psline[linecolor=black, linewidth=0.02](20.27553,2.7386935)(18.833033,-2.7908812)
\psline[linecolor=black, linewidth=0.020657333](16.10393,-1.4123517)(18.839773,-2.7905526)
\psline[linecolor=colour6, linewidth=0.03, linestyle=dashed, dash=0.17638889cm 0.10583334cm](15.3684845,-5.779089)(20.284313,2.7476165)
\psline[linecolor=colour6, linewidth=0.03, linestyle=dashed, dash=0.17638889cm 0.10583334cm](15.36854,-5.755247)(27.354607,-5.755247)
\psline[linecolor=black, linewidth=0.02, arrowsize=0.05291667cm 2.0,arrowlength=1.4,arrowinset=0.0]{->}(18.5919,-0.11845798)(14.142602,-0.868068)
\psline[linecolor=black, linewidth=0.02, arrowsize=0.05291667cm 2.0,arrowlength=1.4,arrowinset=0.0]{<-}(14.094239,-0.868068)(16.270527,2.8799818)
\psline[linecolor=black, linewidth=0.02, arrowsize=0.05291667cm 2.0,arrowlength=1.4,arrowinset=0.0]{->}(20.187843,2.6260817)(22.36413,6.3741317)
\psline[linecolor=black, linewidth=0.02, arrowsize=0.05291667cm 2.0,arrowlength=1.4,arrowinset=0.0]{->}(18.640263,-0.09427698)(16.270525,2.855801)
\psline[linecolor=black, linewidth=0.03, arrowsize=0.05291667cm 2.0,arrowlength=1.4,arrowinset=0.0]{->}(11.099177,2.5849168)(12.309377,4.802627)
\psline[linecolor=black, linewidth=0.02](8.240868,-2.2198353)(8.265048,-4.444484)
\psline[linecolor=black, linewidth=0.02](9.348247,-0.36297163)(10.122038,-2.9261541)
\psline[linecolor=colour6, linewidth=0.03, linestyle=dashed, dash=0.17638889cm 0.10583334cm](7.2482057,-3.906535)(11.091619,2.5942028)
\psline[linecolor=black, linewidth=0.020657333](6.924077,-1.5593172)(9.659923,-2.9375184)
\psline[linecolor=black, linewidth=0.02](11.095678,2.591728)(9.65318,-2.937847)
\psline[linecolor=black, linewidth=0.02](11.095678,2.598799)(6.9237485,-1.5731308)
\psarc[linecolor=black, linewidth=0.018, linestyle=dashed, dash=0.17638889cm 0.10583334cm, dimen=outer, arrowsize=0.05291667cm 2.0,arrowlength=1.4,arrowinset=0.0]{->}(8.234815,-2.2332768){1.2222222}{240.0}{270.0}
\psarc[linecolor=black, linewidth=0.018, linestyle=dashed, dash=0.17638889cm 0.10583334cm, dimen=outer, arrowsize=0.05291667cm 2.0,arrowlength=1.4,arrowinset=0.0]{->}(9.346443,-0.3774628){1.2222222}{240.0}{287.0}
\rput[bl](7.5814815,-4.0621657){{\LARGE $\delta_r$}}
\rput[bl](8.994815,-2.3243878){{\LARGE $\delta_s$}}
\rput[bl](11.391396,4.5316806){{\LARGE $v$}}
\end{pspicture}
}


}
    \caption{Representation of the state vector variable and the inputs $\delta_s$ and $\delta_r$ and representation of the true wind and apparent wind.}
    \label{fig:drawing_boat_ink}
\end{figure}


\begin{equation}
\begin{bmatrix}
\dot{x}\\
\dot{y}\\
\dot{\theta}\\
\dot{v}\\
\dot{\omega}
\end{bmatrix}\  = \begin{bmatrix}
v \cos(\theta)+p_1 a_{tw} \cos(\psi_{tw})\\
v \sin(\theta)+p_1 a_{tw} \sin(\psi_{tw})\\
\omega\\
(g_s \sin(\delta_s)-g_r p_{11} \sin(\delta_r) - p_2 v^2)/p_9\\
(g_s(p_6-p_7\cos(\delta_s))-g_r p_8 \cos(\delta_r)-p_3 \omega v)/p_{10}
\end{bmatrix}
\end{equation}


The states boat is represented by different variables $x$ and $y$ for the position of the boat, a heading $\theta$ 
and an angular speed $\omega$.
The model is non-holonomic such as the boat will always go in the direction $\theta$ (minus the wind drift) therefore the speed $v$ is defined in the boat frame as for the input $[ \delta_s , \delta_r]$, where $\delta_r$ is the rudder angle and $\delta_s$ is the angle of the sail (which is proportional to the sheet length).


$g_s$ and $g_r$ are the force on the sail and on the rudder:
\begin{align}
g_s &= p_4 a_{aw} \sin(\delta_s - \psi_{aw})\\
g_r &= p_5 v^2 \sin(\delta_r)
\end{align}

The force on the sail depend on the apparent wind:

\begin{equation}
\bf{W}_{c,aw}= \begin{bmatrix}
a_{tw} \cos(\psi_{tw} -\theta) - v\\
a_{tw} \sin(\psi_{tw} -\theta)
\end{bmatrix}\
\end{equation}

\begin{equation}
\bf{W}_{p,aw}=\begin{bmatrix}
a_{aw}\\
\psi_{aw}
\end{bmatrix} = \begin{bmatrix}
\mid \bf{W}_{c,aw}\mid\\
\textnormal{angle}( \bf{W}_{c,aw})
\end{bmatrix}\
\end{equation}

$a_{tw}$ is the true wind speed and $\psi_{tw}$ its direction in the world frame
and $a_{aw}$ is the apparent wind speed on the boat and $\psi_{aw}$ the direction of the apparent wind in the boat frame.\\
\begin{minipage}{\linewidth}
\centering
\captionof{table}{Parameters correspondance} \label{tab:title2} 
\begin{center}
\begin{tabular}[t]{|c|l|l|}%{|m{0.10\linewidth}|m{0.3\linewidth}|m{0.1\linewidth}|}
\hline
 $p_1$ & drift coefficient & - \\ \hline
 $p_2$ & tangential friction & $kg \cdot s^{-1}$\\ \hline
 $p_3$ & angular friction & $kg \cdot m$ \\ \hline
 $p_4$ & sail lift & $kg \cdot s^{-1}$ \\ \hline
 $p_5$ & rudder lift & $kg \cdot s^{-1}$ \\ \hline
 $p_6$ & distance to sail & $m$ \\ \hline
 \end{tabular}
 \begin{tabular}[t]{|c|l|l|}%{|m{0.10\linewidth}|m{0.3\linewidth}|m{0.1\linewidth}|}
\hline
 $p_7$ & distance to mast & $m$ \\ \hline
 $p_8$ & distance to rudder & $m$ \\ \hline
 $p_9$ & mass of boat & $kg$ \\ \hline
 $p_{10}$ & moment of inertia & $kg \cdot m^2$ \\ \hline
 $p_{11}$ & rudder break coefficient & - \\ \hline
\end{tabular}
\end{center}
\end{minipage}
\bigskip

The goal of this model is to be simple, not to experience the real reaction of a sailboat but
to be able to implement a controller for this ship. This kind of reasoning has been successful in the past 
with this model and others (see~\cite{LeBars2013}~\cite{Melin2016}).