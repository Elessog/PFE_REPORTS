To test the model, a cable with a pressure sensor has been put on the boat in order to capture the behaviour
of the cable.

First the gathered data will be compared with the old model for tuning. Then the test with cable will be compared to the new model.
Cant expect perfect result because the model of sailboat is not perfect, it only allow to design controller for the boat itself. The goal is the same here determine possibility for controller create a new model for cable (from profile determine from the cable is it doesn't effect to much the boat)
Tests has been realised to test the boat:

\begin{itemize}
\item Cable added on the hull of the boat (between 6 and 20 m)
\item Tapping a pressure sensor on the cable
\item Boat has to reach multiple waypoints while measuring the depth of the pressure sensor
\item Boat redo the same path without cable
\end{itemize}

\section{Validation of cable simulation}

First the depth of the pressure sensor will be compared to the depth of the simulation of the cable when taking the speed and position data.

\begin{figure}[H]
\centering
    \begin{minipage}[b]{0.4\textwidth}
    \centering
    \includegraphics[scale=0.4,angle=0]{depth_over_time_2007_cable}
    \caption{Comparison between depth of cable in simulation and real test.}
    \label{fig:comp_depth_time_2007}
    \end{minipage}
    \hfill
    \begin{minipage}[b]{0.45\textwidth}
    \centering
    \includegraphics[scale=0.4,angle=0]{depth_over_speed_2007_cable}
    \caption{Comparison of depth of cable in function of speed between simulation and real test.}
    \label{fig:comp_depth_speed_2007}
    \end{minipage}
\end{figure}



