\usepackage[top=2.54cm,bottom=2.54cm,left=2.54cm,right=2.54cm,a4paper]{geometry}
\usepackage{xcolor}
\usepackage{hyperref}
\usepackage{url}
\usepackage[utf8]{inputenc} % lettres accentuées
\usepackage[T1]{fontenc}    % Use 8-bit encoding that has 256 glyphs
\usepackage[english]{babel} % Pour le français
\usepackage{graphicx}       % Pour inclure des images
\graphicspath{{images/}}    % Où sont les images ?

\usepackage{listings}      % Pour coloriser les codes que vous insérez
%\lstset{ %
%  backgroundcolor=\color{white},   % choose the background color; you must add \usepackage{color}
%  basicstyle=\footnotesize\ttfamily,        % the size of the fonts that are used for the code
%  breakatwhitespace=false,         % sets if automatic breaks should only happen at whitespace
%  breaklines=true,                 % sets automatic line breaking
%  captionpos=b,                    % sets the caption-position to bottom
%  commentstyle=\color{F18E00},    % comment style
%  deletekeywords={...},            % if you want to delete keywords from the given language
%  escapeinside={\%*}{*)},          % if you want to add LaTeX within your code
%  extendedchars=true,              % lets you use non-ASCII characters; for 8-bits encodings only, does not work with UTF-8
%  %frame=single,                    % adds a frame around the code
%  keepspaces=true,                 % keeps spaces in text, useful for keeping indentation of code (possibly needs columns=flexible)
%  %language=Octave,                 % the language of the code
%  morekeywords={*,...},            % if you want to add more keywords to the set
%  numbers=left,                    % where to put the line-numbers; possible values are (none, left, right)
%  numbersep=8pt,                   % how far the line-numbers are from the code
%  rulecolor=\color{black},         % if not set, the frame-color may be changed on line-breaks within not-black text (e.g. comments (green here))
%  showspaces=false,                % show spaces everywhere adding particular underscores; it overrides 'showstringspaces'
%  showstringspaces=false,          % underline spaces within strings only
%  showtabs=false,                  % show tabs within strings adding particular underscores
%  stepnumber=5,                    % the step between two line-numbers. If it's 1, each line will be numbered
%  tabsize=2,                       % sets default tabsize to 2 spaces
%}
%




\usepackage{booktabs}       % pour de jolis tableaux
%\usepackage{fancyhdr}       % pour des entêtes et pieds de pages améliorés.
\usepackage{hyperref}
\usepackage{makeidx}        % requis pour faire les index
\usepackage{amsmath}
\usepackage{amsfonts}
\usepackage{amssymb}
\usepackage{color}
\usepackage{array}
\usepackage{graphicx}
\usepackage{caption} 
\usepackage{hyperref}
\usepackage{algorithm}
\usepackage{algorithmic}
\usepackage{times}
\usepackage{enumitem}
\usepackage{tabularx}
\usepackage[usenames,dvipsnames]{pstricks}
\usepackage{epsfig}
\usepackage{pst-grad} % For gradients
\usepackage{pst-plot} % For axes
\usepackage[space]{grffile} % For spaces in paths
\usepackage{etoolbox} % For spaces in paths
\makeatletter % For spaces in paths
\patchcmd\Gread@eps{\@inputcheck#1 }{\@inputcheck"#1"\relax}{}{}
\makeatother