The most needing part in this project was the programming of the slam node but the theory behind is approachable , the state equation linking together the poses over the time, and the measurement linking the beacon to the pose.The implemented version of the algorithm is a bit different from the original, with some safeguards and with limitations due to capacity of the current lining of PC (and embeddable PC).

The scan of an area can be done with the created product but not with a good precision for now due to the remaining errors (in programming), therefore the program is more pessimist on the position of the robots and beacons to compensate those errors.\\

What could be done in this project in the future:
\begin{itemize}[label={$-$},itemsep=0cm,topsep=0cm]
\item The correction of the programming errors crippling the program;
\item A re-factoring of the code for a better comprehension from other users
\item An optimization of the code to allow remembering more iteration
\item Consider three dimension
\item Add more path to the controller
\item Make experiment on real robot
\end{itemize}
